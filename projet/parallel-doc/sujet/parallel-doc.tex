\documentclass[12pt,a4paper]{article}
\usepackage[top=1cm, bottom=1.5cm, left=1cm, right=1cm]{geometry}
\usepackage[utf8]{inputenc}
\usepackage[french]{babel}
\usepackage[T1]{fontenc}
\usepackage{CJKutf8}
\usepackage{url}

\begin{document}

\title{Identification d'articles parallèles dans Wikipedia}
\author{Florian Boudin}
\date{Recherche d'information cross-lingue - 2015}

\maketitle

\section{Introduction}

Wikipedia est un projet d'encyclopédie universelle et multilingue (291 langues mi-2015).
%
Lorsqu'un article n'est pas disponible dans la langue souhaitée ou qu'il est incomplet, l'utilisateur peut décider de rechercher l'information dans une autre langue à l'aide des liens inter-langue de Wikipedia.
%
Ces liens sont créés manuellement par les utilisateurs.
%
Par conséquent, de nombreux liens sont manquants et leur mise à jour est une tâche extrèmement chronophage.

\section{Travail demandé}

Votre tâche consiste à développer un système permettant l'identification automatique des liens inter-langue dans Wikipedia.
%
Pour cela, vous disposez de trois ensembles de documents extraits de Wikipedia en trois langues (français, anglais et allemand, \url{http://filex.univ-nantes.fr/get?k=Sqp6QNB4eJhD72SF4qz}).
%
Ces données ont été utilisées dans le cadre de la \textit{shared task} de BUCC 2015~\cite{bucc2015}.


Dans un premier temps, vous devez implémenter la méthode basée sur les hapax présentée dans~\cite{enright2007} (lien direct~: \url{http://www.aclweb.org/anthology/N07-2008.pdf}).
%
Bien que simple, cette méthode donne de bons résultats (30\%+ de MAP).
%
Pour évaluer la performance de votre système, vous utiliserez le logiciel \texttt{trec\_eval} (\url{http://trec.nist.gov/trec_eval/}) et les fichiers de référence disponibles pour le cours (e.g. \texttt{fr-en-train.qrels}).

Dans un second temps, réfléchissez aux moyens d'améliorer cette méthode.
%
Proposez et implémentez vos idées d'amélioration dont vous vérifierez ensuite l'impact sur les données fournies.

\begin{thebibliography}{9}

\bibitem{bucc2015}
  Serge Sharoff, Pierre Zweigenbaum and Reinhard Rapp,
  \emph{BUCC Shared Task: Cross-Language Document Similarity}.
  Proceedings of the Eighth Workshop on Building and Using Comparable Corpora (BUCC),
  2015.

\bibitem{enright2007}
  Jessica Enright and Grzegorz Kondrak,
  \emph{A Fast Method for Parallel Document Identification}.
  Proceedings of the Conference of the North American Chapter of the Association for Computational Linguistics; Companion Volume, Short Papers,
  pages 29--32,
  2007.

\end{thebibliography}

\end{document}  