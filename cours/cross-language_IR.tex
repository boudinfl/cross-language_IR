\documentclass[12pt,aspectratio=43,dvipsnames,table]{beamer}
\usepackage[french]{babel}
\usepackage[T1]{fontenc}
\usepackage[utf8]{inputenc}
\usepackage{url}
\usepackage{multirow}
\usepackage{euler}
% \usepackage{color}

\usetheme{default}
\useinnertheme{default}
\useoutertheme{default}
\usefonttheme{serif}
\usecolortheme[named=WildStrawberry]{structure}

% Marges autour des slides
\setbeamersize{text margin left=5mm, text margin right=5mm}

% Suppression des symboles de navigation
\beamertemplatenavigationsymbolsempty

% Définition du pied de page
\setbeamertemplate{footline} {
  \hfill{
    \insertshortdate ~- %
    page \insertframenumber~sur~\inserttotalframenumber %
    %\hspace*{0.1cm}
  } \vspace*{0.05cm}
}

% Définition des espacements des listes
\setlength{\leftmargini}{0.6cm}
\setlength{\leftmarginii}{0.4cm}
\setlength{\leftmarginiii}{0.4cm}

\title{Recherche d'information cross-lingue}
\subtitle{Applications multilingues - module X9IT100}
\author{Florian Boudin}
\institute{Département informatique, Université de Nantes}
\date[30 juillet 2013 / Rév.~1]{Révision~1 du 30 juillet 2012}

\begin{document}


%-B--------------------------------------------------------------------------B-%
% frame titre
\frame[plain]{\titlepage}
%-E--------------------------------------------------------------------------E-%


%-B--------------------------------------------------------------------------B-%
\begin{frame}
    \frametitle{Préface}
    \begin{itemize} \itemsep10pt
        \item Notions abordées dans ce module
        \begin{itemize}
            \item Rappel des notions de recherche d'information
            \item CLIR
        \end{itemize}
        \item Volume horaire (2h40)
        \item Cours basé sur
        \begin{itemize}
            \item \cite{DBLP:series/synthesis/2010Nie} Jian-Yun Nie, 
                  \textit{Cross-Language Information Retrieval}
        \end{itemize}
    \end{itemize}
\end{frame}
%-E--------------------------------------------------------------------------E-%


%-B--------------------------------------------------------------------------B-%
\begin{frame}
    \frametitle{Introduction}
    \begin{itemize} \itemsep10pt
        \item La Recherche d'Information (RI) fait partie de notre vie 
              quotidienne.
        \item Dans la plupart des cas, nous recherchons des documents rédigés 
              dans notre langue maternelle, en général la langue utilisée dans
              la requête.
        \item Cependant, l'information pertinente n'est pas toujours disponible
              dans notre langue maternelle, et le web offre une mine 
              d'information riche et multilingue à laquelle nous souhaitons 
              avoir accès.
        \item \'Emergence de la problématique de la RI cross-lingue
    \end{itemize}
\end{frame}
%-E--------------------------------------------------------------------------E-%

%-B--------------------------------------------------------------------------B-%
\begin{frame}

    \frametitle{}

\end{frame}
%-E--------------------------------------------------------------------------E-%

% %-B------------------------------PLAN----------------------------------------B-%
% \begin{frame}
% \frametitle{Plan}
% \tableofcontents[sectionstyle=show,subsectionstyle=hide,subsubsectionstyle=hide]
% \end{frame}
% %-E--------------------------------------------------------------------------E-%

%-B--------------------------------------------------------------------------B-%
\begin{frame}
    \frametitle{References}
    \bibliographystyle{alpha}
    \bibliography{bibliography}
\end{frame}
%-E--------------------------------------------------------------------------E-%


\end{document}